\heada{SOLUTION}{TOLerance}
\hspace{1.2cm}{{tol,,v1 \hfill}} \\{\smallskip}
\hspace{1.0cm}{{tol,ener,v1 \hfill}} \\{\smallskip}
\hspace{1.0cm}{{tol,emax,v1 \hfill}}
\headb

The {\tt TOL} command is used to specify the solution tolerance
values to be used at various stages in the analysis.  Uses include:
\begin{enumerate}
\item{
Convergence of nonlinear problems in terms of the norm
of  energy in the current iterate (the inner, dot, product
of the displacement  increment  and  the  solution
residual vectors).}
\item{
Convergence of the subspace eigenpair solution which is
measured  in  terms  of the change in subsequent eigenvalues computed.}
\end{enumerate}

The default value of {\tt TOL} is 1.0d-16.

The tol command also permits setting a value for the energy
below which convergence is assumed to occur.  The command is issued as
{\tt TOL,ENER}gy,{\tt v1} where {\tt v1} is the value of the converged
energy (i.e., it is equivalent to the tolerance times the maxiumum energy
value).  Normally, {\sl FEAPpv} performs nonlinear iterations until the
value of the energy is less than the {\tt TOL}erance value times the
value of the energy from the first iteration.  However, for some 
transient problems the value of the initial energy is approaching
zero (e.g., for highly damped solutions which are converging to some
steady state limit).  In this case, it is useful to specify the energy
for convergence relative to early time steps in the solution.  
Convergence will be assumed if either the normal convergence criteria
or the one relative to the specified maximum energy is satisfied.

Finally, the tol command permits resetting the maximum energy value used
for convergence.  The command is issued as
{\tt TOL,EMAX}imum,{\tt v1} where {\tt v1} is the value of the maximum
energy quantity. Since the {\tt TIME} command sets the maximum energy to
zero, the value of {\tt EMAX}imum must be reset after each time step. Thus,
a set of commands:
\begin{verbatim}
       LOOP,time,n
         TIME
         TOL,EMAX,5.e+3
         LOOP,newton,m
           TANG,,1
         NEXT
         etc.
       NEXT
\end{verbatim}
is necessary to force convergence check against a specified maximum energy.
The above two forms for setting the convergence are nearly equivalent; however,
the {\tt ENER}gy tolerance form can be set once whereas the {\tt EMAX}imum
form must be reset after each time command.
\vfill\eject
