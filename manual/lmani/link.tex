\headm{MESH}{LINK}
\hspace{1.0cm}{{ link \hfill}}\\{\smallskip}
\hspace{1.4cm}{{ node1,node2,inc1,inc2,(idl(i),i=1,ndf) \hfill}}\\{\smallskip}
\hspace{1.4cm}{{ <terminate with a blank record> \hfill}}
\headb

A mesh may be generated in {\sl FEAPpv} in which it is desired
to let the some or all of the degree of freedom values at
more than one node share the same displacement unknown.
For example, in repeating structures the value of the dependent
variable will be the same at each repeating interval.
In a finite element model it is necessary to specify the
repeating condition by linking the degree of freedoms at
theses nodes to the same unknown in the equations.  The
{\tt LINK} command is used for this purpose.

To use the {\tt LINK} option the complete mesh must first
be defined.  After the {\tt END} command for the mesh definition
and before the {\tt BATC}h or {\tt INTE}ractive command
for defining a solution algorithm, the use of a {\tt LINK}
statement together with the list of affected nodes and
degree of freedoms will cause the program to search for all
conditions that are to be connected together.

The input data is interpretted as follows:

\begin{center}
\begin{tabular}{r l}
\it  node1   &-- Node on one body to be linked           \\
\it  node2   &-- Node on other body to be linked           \\
\it  inc1    &-- Increment to generate additional nodes     \\
             &\quad for node1           \\
\it  inc2    &-- Increment to generate additional nodes     \\
             &\quad for node2           \\
\it  idl(1)  &-- Linking condition, 0 = link, non-zero do   \\
             &\quad not link dof 1.           \\
\it  idl(2)  &-- Linking condition, 0 = link, non-zero do   \\
             &\quad not link dof 2.           \\
             &\quad etc. for {\it ndf} degree of freedoms      \\
\end{tabular}
\end{center}
Generation is accomplished by giving a pair of  records.   A
generation  terminates  whenever  one  of  the  sequences is
reached.  For example:

\begin{verbatim}
       LINK
         5,105,3,5,1,0,1
         15,140,,,1,1,0}
\end{verbatim}
will generate the sequence of links
\begin{center}
\begin{tabular}{r|r|c}
Node 1& Node 2& Link Codes \\ \hline
 5& 105& 1 0 1 \\
 8& 110& 1 0 1 \\
11& 115& 1 0 1 \\
14& 120& 1 0 1 \\
15& 140& 1 1 0
\end{tabular}
\end{center}
Termination of input occurs with a blank record.

Whenever it is desired to only connect {\it node1} to
{\it node2}, {\it inc1} and {\it inc2} need not be specified (they may
be blank or zero).
\vfil\eject
