\heada{SOLUTION}{UTANgent}
\hspace{1.2cm}{{ utan,,<n1,v2> \hfill}}\\{\smallskip}
\hspace{1.0cm}{{ utan,line,<n1,v2,v3> \hfill}}
\headb

The {\tt UTAN}gent command computes an unsymmetric tangent stiffness matrix
about the current value of the solution state vector.  For
linear applications the current stiffness matrix is just the
normal stiffness matrix.

If the value of {\tt n1} is non-zero, a force vector for
the current residual is also computed (this is identical to
the {\tt FORM} command computation) - thus leading to greater
efficiency when both the tangent stiffness and a residual
force vector are needed.

If the value of {\tt v2} is non-zero a {\it shift} is applied
to the stiffness matrix in which the element mass matrix is
multiplied by {\tt v2} and subtracted from the stiffness matrix.
This option may not be used with the {\tt SUBS}pace algorithm,
which is restricted to symmetric tangents only (see {\tt TANG}ent.
The shift may be used to represent a forced vibration
solution in which all loads are assumed to be harmonic
at a value of the square-root of {\tt v2} (rad/time-unit).

After the tangent matrix is computed, a triangular
decomposition is available for subsequent solutions using
{\tt FORM} and {\tt SOLV}e, etc.

In the solution of non-linear problems, using a full or
modified Newton method, convergence from any starting point
is not guaranteed.  Two options exist within available
commands to improve chances for convergence.  One is to use
a line search to prevent solutions from diverging rapidly.
Specification of the command {\tt UTAN,LINE} plus options
invokes the line search option (it may also be used in conjunction
with {\tt SOLVe,LINE} in modified Newton schemes).  The
parameter {\tt v3} is typically chosen between 0.5 and 0.8
(default is 0.8).

The second option to improve convergence of non-linear
problems is to reduce the size of the load step increments.
The command {\tt BACK} may be used to {\it back-up} to the
beginning of the last time step (all data in the solution
vectors is reset and the history data base for inelastic
elements is restored to the initial state when the current
time is started). Repeated use of the {\tt BACK} command may be
used. However, it applies only to the current time interval.
The loads may then be adjusted and a new solution with
smaller step sizes started.

The {\tt UTAN}gent operation is normally the most time consuming
step in problem solutions - for large problems several
seconds are required - be patient!
\vfill\eject

