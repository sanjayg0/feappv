\section{Multiple NURBS blocks}

Problems may be solved using multiple NURBS blocks.  However, the edge or
boundary of contiguous blocks must have the same control point topology --
the topology in other directions may be different.

If multiple blocks are used, prior to any solution commands  it is
necessary to merge the blocks into a single
problem using the \texttt{TIE} command.
This command appears after the \texttt{END} of mesh command, thus,
the general form is:
\begin{verbatim}
       FEAP * * <title information
         0 0 ....
        <mesh data>
       END ! End of mesh
       TIE
       <solution commands>
\end{verbatim}

\subsection{Slope compatibility enforcement for thin shells: NTIE}
\label{ntie}

For the thin shell element use of the \texttt{TIE} command results in
a moment-less hinge between the patches.  In order to restore slope
compatibility between patches it is necessary to add the command
\texttt{NTIE} after the \texttt{TIE} command.
It is possible enforce slope compatibility between specific patches or
to enforce it between all shell patches.
To enforce a compatibility between individual patches the command is
given as
\begin{verbatim}
       TIE
       NTIE
         PATCh p1 p2 ma
\end{verbatim}
where \texttt{p1} and \texttt{p2} are the two patches and
\texttt{ma} is the material set number for the
\texttt{NURB\_SLOPE} material set (see Section \ref{nmaterial}).

To enforce compatibility between all patches the command 
is given as
\begin{verbatim}
       TIE
       NTIE
         ALL ma
\end{verbatim}
where \texttt{ma} is the material set number for the
\texttt{NURB\_SLOPE} material set (see Section \ref{nmaterial}).

\section{OUTMesh: Output of IgA mesh}

Once a final mesh for a problem is created the data sets may be 
saved to a file using the command:
\begin{verbatim}
       OUTMesh
\end{verbatim}
The output is written to either a file \texttt{Ixxxx.rev} if no profile 
oprimization was specified or \texttt{Ixxxx.opt} if profile optimization was
specified prior tothe \texttt{OUTMesh} command.  The character string
\texttt{xxxx} contains the name of the file initially specified in the
IgA \textsl{FEAP} run.  The written file contains the following input data sets:
\begin{verbatim}
       MATErials  ! any material sets specified
       NURBS ALL  ! with coordinates and weights for each node
       ELEMents   ! with nodal connections for each element
       ENURBs     ! a data set defining location and number of knots
       KNOTs      ! with list of knot vector data
        ....      ! any specified loads, displacements, and/or b.c.
\end{verbatim}
The only solution command contained in the file is \texttt{INTEractive}.

\section{Surface extraction}

In many problems it is necessary to define segments of surfaces from the
NURBS patches. These may be found using the solution command \texttt{N\_EXtract}.
This should be performed in an \textit{interactive mode of solutions}.
To initiate the extraction it is necessary to first display a plot of the
problem in graphics mode.  For two dimensional problems one should first
give the command
\begin{verbatim}
       PLOT MESH
\end{verbatim}
this is then followed with the command
\begin{verbatim}
       N_EXtract
\end{verbatim}
The program will then display each of the boundary segments for each NURBS
patch and the user may select to output a file or reject it.

In a three dimensional problem the graphics commands
\begin{verbatim}
       PLOT PERSpecitive
       PLOT HIDE
       PLOT MESH
\end{verbatim}
should be issued prior to giving the
\begin{verbatim}
       N_EXtract
\end{verbatim}
command.

After completing the selection of any boundary segments a set of files containing the \texttt{ELEMents} and \texttt{ENURbs} data will be created.  A single
file
\begin{verbatim}
       Bxxx_m
\end{verbatim}
where \texttt{xxx} are the characters (3:5) of the input file name, contains
a record
\begin{verbatim}
       *auto
\end{verbatim}
and a list of the files containing the surface segment extractions.
These should be reviewed to ensure the created mesh data is correct.
In particular the material set number for each segment.
The basic header to change is
\begin{verbatim}
       ELEMent NODE=no_nd MATE=ma TYPE ....
\end{verbatim}
The material number for the entire set can be changed by setting the value of
\texttt{ma} desired.  \textit{It is not necessary to change the number on
each element data record}.  Do not edit any data in the date part
\texttt{ENURbs}.  This is used to select the correct extraction operator
for each element.

The file \texttt{Bxxx\_m} should be added to the mesh part of the input
file \textit{that created the boundary segments} as
\begin{verbatim}
       include Bxxx_m
\end{verbatim}

\section{Problem solution}

After the problem is formed, standard \textsl{FEAP} solution commands are
used to solve each problem.

\section{Graphics outputs}

Standard \textsl{FEAP} plot commands may be used to display the location of the
control points (\texttt{PLOT NODE} command) and boundary restraints on the
control points (\texttt{PLOT BOUNdary}).  Use of any contouring
commands (\texttt{PLOT CONT}, \texttt{PLOT STRE}) is performed by projection
onto 4-node quadrilateral sub-elements of the surface.  Most plot commands may be
used but there remain some delicate transformations between the various representation of the data to be plotted.  For 3-d objects plots should be given
in perspective mode.  Thus, the data for each plot sequence should begin with
\begin{verbatim}
       PLOT PERSpective  ! Required for 3-d only
       PLOT HIDE         ! Required for 3-d only
\end{verbatim}

If a problem requires long solution times it is advisable to use \texttt{SAVE}
commands to preserve solution values prior to attempting plot outputs.
