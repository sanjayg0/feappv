\heada{MESH}{FEAPpv}
\hspace{1.0cm} {{ feappv [ title of problem for printouts, etc.]\hfill}}
\\{\smallskip}
\hspace{1.2cm} {{ numnp,numel,nummat,ndm,ndf,nen,npd,nud,nad \hfill}}
\headb

Each problem to be solved by {\sl FEAPpv} initiates with a single record
which contains the characters {\tt FEAPpv} (either in upper or
lower case) as the first
entry; the remainder of the record (columns 5-80) may be used to
specify a problem title.  The title will be printed
as the first line of output on each page.
The {\tt FEAPpv} record may be preceded by {\tt PARA}meter specifications
(see parameter input manual page).

Immediately following the {\tt FEAPpv} record the {\it control}
information describing characteristics of the finite element
problem to be solved must be given.
The {\it control} information data entries are:

\begin{center}
\begin{tabular}{r l}
\it  numnp  &-- Total number of nodal points in the problem. \\
\it  numel  &-- Total number of elements in the problem. \\
\it  nummat &-- Number of material property sets in the problem. \\
\it  ndm    &-- Number of spatial coordinates needed to define mesh. \\
\it  ndf    &-- Maximum number of degrees-of-freedom on any node. \\
\it  nen    &-- Maximum number of nodes on any element. \\
\it  npd    &-- Maximum number of parameters for element properties. \\
            &   \quad (default 200). \\
\it  nud    &-- Maximum number of parameters for user element properties. \\
            &   \quad (default 50). \\
\it  nad    &-- Increases size of element arrays to {\it ndf$\times$nen+nad}. \\
\end{tabular} 
\end{center}
For many problems it is not necessary to specify values for {\it numnp},
{\it numel}, or {\it nummat}.  {\sl FEAPpv} can compute the maximum values for
each of these quantities.  However, for some meshes or when user functions
are used to perform the inputs it is necessary to assign the values for
these parameters.

The number of spatial coordinates needed to define the finite
element mesh {\it (ndm)} must be 1, 2, or 3.  The maximum
number of the other quantities is limited only by the size of the
dynamically dimensioned array used to store all the data and solution
parameters.  This is generally quite large and, normally, should not
be exceeded.  If the error message that memory is exceeded appears the
data should be checked to make sure that no errors exist which could
cause large amounts of memory to solve the problem. For example, if the error
occurs when the {\tt TANG}ent or {\tt UTAN}gent
solution macro statements
are encountered, the profile of the matrix should be checked for very
large column heights (can be plotted using the {\tt PLOT,PROF}ile command).
Appropriate renumbering of the mesh or use of the
solution command {\tt OPTI}mize can
often significantly reduce the storage required.
For symmetric tangent problems the use of the sparse solution routine,
which invoked using the solution command {\tt DIREct,SPAR}se, often requires
significantly less memory.  For some problems with symmetric tangents
a solution can be achieved using the iterative conjugate gradient solution
method (invoked by the {\tt ITER}ation solution command.

If necessary, the
main subprogram, {\tt program feappv}, can be recompiled with a larger
value set for the {\tt parameter mmax} controling the size of blank common.
\vfil\eject
