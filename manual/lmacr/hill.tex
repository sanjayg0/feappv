\heada{SOLUTION}{HILL}
\hspace{1.0cm}{{hill  \hfill}}
\hspace{1.2cm}{{ hill,stress \hfill}} \\{\smallskip}
\hspace{1.2cm}{{ hill,tangent \hfill}}
\headb

\index{Solution command!HILL}
The command {\tt HILL} is used to perform computation of averaged properties
on a representative volume element (RVE) using the Hill-Mandel hommogenization
method.  The approach may be used to compute the properties for either
a thermal problem and a small or finite deformation solid problem.
For this case a typical RVE may be defined and subjected to
displacement boundary
conditions. The boundaries of the RVE are subjected to a homogeneous 
thermal gradient: a strain
(for small deformations) or a deformation gradient (for finite deformations).
The description for the gradients is given in Sect. \ref{periodicbc}.

The computation of the averaged properties is determined using a solution
command
\begin{verbatim}
       LOOP,,nits
         TANGent,,1
       NEXT
       HILL <TANGent,     >
\end{verbatim}
to obtain the homogenized stress (or thermal flux) and associated tangent array
or
\begin{verbatim}
       LOOP,,nits
         TANGent,,1
       NEXT
       HILL STREss
\end{verbatim}
to obtain stress or thermal flux averages only. 
THe \texttt{LOOP-NEXT} is not needed for linear problems.
The properties for non-linear problems subjected to time varying loads may
also be obtained, however, in this case it is necessary to define
time variations to the deformation by a proportional loading.
\vfill\eject
