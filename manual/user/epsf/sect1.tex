\chapter[Introduction]{INTRODUCTION}
\label{intro}

During the last several years, the finite element method has evolved
from a linear structural analysis procedure to a general technique
for solving non-linear partial differential equations.  An extensive
literature exists on the method describing
the theory necessary to formulate solutions
to general classes of problems. It is
assumed that the reader is familiar with the finite element method as
describe in popular reference books (e.g., {\it The Finite Element Method},
4th edition, by O.C. Zienkiewicz and R.L. Taylor ~\cite{zt1,zt2} or
the 5th edition ~\cite{zt1n,zt2n,zt3n}) and
desires either to solve a specific problem or to
generate new solution capabilities.

The Finite Element Analysis Program
({\sl FEAP}) is a computer analysis system designed for:

\begin{enumerate}
\item Use in an instructional program to illustrate performance
of different types of elements and modeling methods; 
\item In a research, and/or applications
environment which requires frequent modifications to address new
problem areas or analysis requirements.
\end{enumerate}
The computer system has been
developed primarily for UNIX work station and personal computer (PC)
environments
and includes an integrated set of modules to perform input of data describing
a finite element model, construction of solution algorithms to address a wide
range of applications, and graphical and numerical output of solution results.

A problem solution is constructed using a command
language concept in which the solution algorithm is
written by the user.  Accordingly, with this capability, each
user may define a solution strategy which meets specific needs.
There are sufficient commands included in the system for
applications in structural or fluid mechanics, heat transfer, and
many other areas requiring solution of problems modeled by differential
equations; including those
for both steady state and transient problems.

Users also may add new features for model description and command language
statements to meet specific applications requirements. These additions 
may be used to assist users in generating meshes
for specific classes of problems or to import meshes generated by other
systems.

The {\sl FEAP} system includes a general element library.
Elements are available to model one, two and three dimensional problems
in linear and non-linear
structural and solid mechanics and for linear heat conduction problems.
Each element uses a material
model library. Material models are provided for elasticity,
viscoelasticity, plasticity, and heat transfer constitutive equations.
Elements also provide capability to generate mass and
geometric stiffness matrices for structural problems and to compute output
quantities associated with each element (e.g., stress, strain), including
capability of projecting these quantities to nodes to permit graphical
outputs of result contours.

Users also may add an element to the system by writing and linking a single
module to the {\sl FEAP} system. Details on
specific requirements to add an element as well as other optional
features available are included in of the {\sl FEAP Programmers
Manual}.

The next several sections of this manual describe how to use existing
capabilities in the {\sl FEAP} system.  In the next several chapters
the general features of {\sl FEAP} are described.  The discussion
centers on three different phases of problem solution using {\sl FEAP}:
\begin{enumerate}
\item Mesh description options;
\item Problem solution options; and
\item Graphical display options.
\end{enumerate}
The {\sl FEAP Example Manual}
may be consulted for examples of some of the input and solution
options available.

\section{Changes in Recent and Current Releases}

The release \textsl{FEAP} Version 7.1 introduced some new capabilities
and revises the way commands perform from previous versions.

In the data input the main
change is the manner in which the boundary condition data is processed.
It is highly recommended that users carefully read Chapter \ref{bc} concerning
these changes.  Specifically, the way in which data is either {\it set} or
{\it accumulated} has been revised significantly.

The ability to add Rayleigh damping has been extended to transient solution
by time integration methods as well as using modal methods.  This feature
is limited to the small deformation solid and structural elements in the
Version 7.1 release; however, it may be used with any of the material models
available in the system.

The discrete elements for lumped mass has been modified
to permit the specification of proportional acceleration loadings, such
as those specified for earthquake ground motions.  Proportional factors
are included as global data and a new command to specify the appropriate
proportional loading has been added.

Starting with version 7.1, new features to treat meshes consisting of
three dimensional tetrahedral elements
have been added.  This is to facilitate the use of {\sl FEAP} with automatic
mesh generation programs which commonly generate only tetrahedral elements.
The graphics part can also display tetrahedral elements.  At present only
the {\tt SOLId} displacement and {\tt THERmal} formulations
can treat tetrahedral elements.
Thus, solution of any problem in which nearly incompressible
behavior may be encountered should not be attempted without adding a user
element designed for this purpose.
 
The transient integrators are modified to permit a specification of the 
{\it order} of the time operator.  This permits the solution of coupled
problems in which the order may be different between the unknowns (e.g.,
as in thermo-mechanical problems).

In the definition of solution commands several new features have been
added.  These include:
More general treatment of time dependent outputs using {\tt TPLOt}
are provided; alternative forms for using {\tt PRINt} and {\tt NOPRint}
during solution; an option to include or exclude a {\it geometric}
stiffness during solution iterations; to name a few.

The Version 7.1 release also included a capability to solve contact problems
in two and three dimensions.  The definition of the contact surfaces
and the mode of interaction is defined in a new section of the manual,
Chapter \ref{contact}.  The solution of contact problems is one of the
most difficult in transient finite deformation solid mechanics.  The
current release can treat both frictional and frictionless problems for
some problems.  However, it is expected that continued improvements will
be required to treat general classes of problems.

In \textsl{FEAP} Version 7.2 numerous small changes have been incorporated
which correct errors or other aspects of solution.  For example, when
requesting output data for use in constructing time history plots using
the \texttt{PLOT TPLOt} command the results for the last time step are
automatically added to the file when execution ceases.  In two dimensional
problems where surface tractions are specified using the \texttt{CSURface}
command, the end points need not be placed at or near a node.  They may
be at arbitrary points along the surface.
PostScript output of color line plots from truss and frame elements
may be obtained and color changes are correctly obtained.
Finally, the ability of graphically displaying results in a cylindrical
coordinate frame has been included in the current release.  Thus, it is
possible to display either the cartesian components or the cylindrical
coordinat components for displacement (\texttt{CONTour}), velocity
(\texttt{VELOcity}), acceleration (\texttt{ACCEleration)}, stress
(\texttt{STREss} or \texttt{ESTRE}) plot commands.

The \texttt{GAP} element has added capabililty.
It is no longer necessary to have the element pointed in
a positive coordinate direction (i.e., with node 1 to 2 considered in
a positive coordinate sense) -- negative directions may be used.  Also,
the degree of freedom to use for the gap condition may be other than the
one in the coordinate sense.  Details are given in Chapter \ref{elmlib}.

Major changes introduced into Version 7.2 include an option to use Lagrange
multiplier methods to impose contact constraints.  These may be combined
with penalty type regularization to give many new algorithmic options to
solve this class of problems.  In the input phase it is now possible to
use \texttt{LOOP}-\texttt{NEXT} constructs for replicating parts which
are identical or similar except for coordinate transformations.  Use of
the looping feature is described in Section \ref{mloop}.  This
feature is new and will undoubtably have some aspects which do not fully
function.  Users should report any problems so that corrections and extensions
may be made.


Other changes include additional output to the screen indicating the
results from specified commands.
