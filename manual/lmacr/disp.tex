\heada{SOLUTION}{DISPlacements}
\hspace{1.2cm} {{ disp,,<n1,n2,n3> \hfill}} \\{\smallskip}
\hspace{1.0cm} {{ disp,coor,idir,xi \hfill}} \\{\smallskip}
\hspace{1.0cm} {{ disp,all \hfill}} \\{\smallskip}
\hspace{1.0cm} {{ disp,eigv,<n1,n2,n3> \hfill}}
\headb

The command {\tt DISP}lacement may be used to print the
current values of the solution vector as follows:

\begin{enumerate}
\item{
Using the command:

\begin{verbatim}
          disp,,n1,n2,n3
\end{verbatim}
prints out the current solution vector for nodes
{\tt n1} to {\tt n2} at increments of {\tt n3} (default increment = 1).  If
{\tt n2} is not specified only the value of node {\tt n1} is
output.  If both {\tt n1} and {\tt n2} are not specified only
the first nodal solution is reported.}


\item{
If the command is specified as:

\begin{verbatim}
          disp,coor,idir,xi
\end{verbatim}
all nodal quantities for the coordinate direction {\tt idir}
with value equal to {\tt xi} are output.

Example: 
\begin{verbatim}
          disp,coor,1,3.5
\end{verbatim}
prints all the nodal solution vector which have $x_1$ = 3.5.

This is useful to find the nodal values along a particular
constant coordinate line.}

\item{
If the command is specified as:

\begin{verbatim}
          disp,all
\end{verbatim}
all nodal solutions are output.}
\end{enumerate}

In order to output a solution vector it is first necessary to
specify commands language instructions to compute the desired
values, e.g., for displacements perform a static or transient analysis.
\vfill\eject
