\heada{SOLUTION}{IF}
\hspace{1.0cm}{{if expression \hfill}}
\headb

\index{Solution command!IF}
\index{IF-ELSE-ENDIf control}
The \texttt{IF} command must be used with a matching \texttt{ENDI}f command.
Optionally, one or more \texttt{ELSE} commands may be included between the
\texttt{IF-ENDI}f pair.
The \texttt{expression} is used to control the actions taken during the
solution.  If the expression evaluates to be positive then the commands
contained between the \texttt{IF} and the \texttt{ELSE} or \texttt{ENDI} are 
executed, otherwise solution continues with a check of the next \texttt{ELSE}
For example, the sequence

\begin{verbatim}
        ...
       IF  10-a
         tang,,1
         param a 0
       ELSE
         form
         solv
       ENDIf
       param a = a+1 
        ...
\end{verbatim}
\par\noindent
would compute a tangent, residual, and solution increment if \texttt{10-a}
is positive; otherwise the solution increment is computed using a
previous tangent.  The parameter \texttt{a} is computed using a \texttt{PARAM}
command.
\vfill\eject
