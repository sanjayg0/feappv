\heada{SOLUTION}{EIGVectors}
\hspace{1.2cm} {{ eigv,nn,<n1,n2,n3> \hfill}} \\{\smallskip}
\hspace{1.0cm} {{ eigv,coor,idir,xi,nn \hfill}} \\{\smallskip}
\hspace{1.0cm} {{ eigv,all,n1,nn \hfill}} \\{\smallskip}
\headb

The command {\tt EIGV}ector may be used to print the
current values of eigenvector vector \texttt{nn} as follows:

\begin{enumerate}
\item{
Using the command:

\begin{verbatim}
          eigv,nn,n1,n2,n3
\end{verbatim}
prints out the eigenvector \texttt{nn} for nodes
{\tt n1} to {\tt n2} at increments of {\tt n3} (default increment = 1).  If
{\tt n2} is not specified only the value of node {\tt n1} is
output.  If both {\tt n1} and {\tt n2} are not specified only
the first nodal solution is reported.}

\item{
If the command is specified as:

\begin{verbatim}
          eigv,coor,idir,xi,nn
\end{verbatim}
all nodal quantities for the coordinate direction {\tt idir}
with value equal to {\tt xi} are output.

Example: 
\begin{verbatim}
          eigv,coor,1,3.5,2
\end{verbatim}
prints all the nodes in eigenvector 2 which have $x_1$ = 3.5.

This is useful to find the nodal values along a particular
constant coordinate line.}

\item{
If the command is specified as:

\begin{verbatim}
          eigv,all,nn
\end{verbatim}
all nodal solutions for eigenvector \texttt{nn} are output.}
\end{enumerate}

In order to output a solution vector it is first necessary to
specify commands language instructions to compute the desired
values, e.g., for displacements perform a static or transient analysis.

\vfill\eject
