\section{Regions and Element Groups}
\label{regions}

The elements in {\sl FEAP} may be assigned to different groups using the
{\tt REGI}\-on command.  The command is given as
\begin{verbatim}
       REGIon,number
\end{verbatim}
where {\tt number} is an integer constant of parameter defining the group
number for the elements.
Any elements which are input after a region command is given belong
to the given group number.
By default all elements are assigned to region zero.

The use of regions facilitates two aspects in {\sl FEAP}.  The first is
for use in merging groups of elements whose nodes should be common
but have different numbers (e.g., those defined using {\tt BLOC}k commands).
An illustration of this option is used in Example 4 in the {\sl Example Manual}.
The second use is to activate or deactivate elements to represent excavation
or construction sequences.  This option uses the {\tt ACTI}vate or
{\tt DEAC}tivate command language instructions (see Appendix B).
Elements in Region zero may not be deactivated.

\section{Flexible or Rigid Groups}
\label{flexrigid}

{\sl FEAP} permits the use of both flexible and rigid finite elements.
By default all elements are flexible.  If it is desired to designate an
element as rigid the command
\begin{verbatim}
       RIGId,number
\end{verbatim}
must be inserted in the mesh data just before the elements belonging to the
rigid body {\tt number} are input or generated using the {\tt ELEM}ent,
{\tt BLOC}k, or {\tt BLEN}d commands.

To designate elements as flexible the command
\begin{verbatim}
       FLEXible
\end{verbatim}
must be inserted immediately before element groups which are to remain
deformable.
It is not necessary to include this statement if all elements are flexible.

{\it The current release of} {\sl FEAP} {\it does not fully support
rigid body options.  Problems may be solved using the energy conserving
algorithms; however, other algorithms may not converge quickly.}
