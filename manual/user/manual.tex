% FEAPpv Manual: Version 4.1 - Revised: September 2014

\documentclass[12pt]{report}

\topmargin 0.4in
\advance \topmargin by -\headheight
\advance \topmargin by -\headsep

\setlength{\textwidth}{15.5cm}
\setlength{\oddsidemargin}{+0.4cm}
\setlength{\evensidemargin}{+0.4cm}
\setlength{\textheight}{21.7cm}
\setlength{\parindent}{0pt}
\setlength{\parskip}{3mm}

\usepackage[colorlinks=true,citecolor=blue,linkcolor=blue,bookmarksnumbered=true,pdfstartview={FitH},hyperindex=true,linktoc=all,linktocpage=true]{hyperref}

\usepackage{amsmath}
\usepackage{graphicx}

\newcommand{\Bf}[1]{\boldsymbol{#1}}
\newcommand{\B}[1]{\mathbf{#1}}
\newcommand{\ul}{\underline}
\newcommand{\scite}[1]{\textsuperscript{\cite{#1}}}
\newcommand{\q}{\quad}
\newcommand{\qq}{\qquad}
\newcommand{\pdif}[2]{\frac{\partial {#1}}{\partial {#2}}}
\newcommand{\odif}[2]{\frac{\dif {#1}}{\dif {#2}}}
\newcommand{\dpdif}[2]{\dfrac{\partial {#1}}{\partial {#2}}}
\newcommand{\dodif}[2]{\dfrac{\dif {#1}}{\dif {#2}}}
\newcommand{\ph}{\phantom}
\font\babc=cmmib10 at 10.5pt\def\balpha{\hbox{\babc\char'013}}
\font\baabc=cmmib10 at 10.5pt\def\bbeta{\hbox{\baabc\char'014}}
\font\babbc=cmmib10 at 10.5pt\def\bgamma{\hbox{\babbc\char'015}}
\font\babcc=cmmib10 at 10.5pt\def\bdelta{\hbox{\babcc\char'016}}
\font\bacd=cmmib10 at 10.5pt\def\bepsilon{\hbox{\bacd\char'42}}
\font\bacdd=cmmib10 at 10.5pt\def\bomega{\hbox{\bacdd\char'41}}
\font\bacdf=cmmib10 at 10.5pt\def\bzeta{\hbox{\bacdf\char'20}}
\font\bafa=cmmib10 at 10.5pt\def\bfeta{\hbox{\bafa\char'21}}
\font\bamn=cmmib10 at 10.5pt\def\btheta{\hbox{\bamn\char'22}}
\font\banm=cmmib10 at 10.5pt\def\bkappa{\hbox{\banm\char'24}}
\font\bapk=cmmib10 at 10.5pt\def\blambda{\hbox{\bapk\char'25}}
\font\bappk=cmmib10 at 10.5pt\def\bmu{\hbox{\bappk\char'023}}
\font\balk=cmmib10 at 10.5pt\def\bnu{\hbox{\balk\char'27}}
\font\bauk=cmmib10 at 10.5pt\def\bxi{\hbox{\bauk\char'30}}
\font\bauj=cmmib10 at 10.5pt\def\bmu{\hbox{\bauj\char'26}}
\font\baty=cmmib10 at 10.5pt\def\brho{\hbox{\baty\char'32}}
\font\bayh=cmmib10 at 10.5pt\def\bsigma{\hbox{\bayh\char'33}}
\font\bakh=cmmib10 at 10.5pt\def\btau{\hbox{\bakh\char'34}}
\font\baik=cmmib10 at 10.5pt\def\bupsilon{\hbox{\baik\char'35}}
\font\bati=cmmib10 at 10.5pt\def\bphi{\hbox{\bati\char'36}}
\font\baio=cmmib10 at 10.5pt\def\bchi{\hbox{\baio\char'37}}
\font\blpi=cmmib10 at 10.5pt\def\bpsi{\hbox{\blpi\char'40}}
\font\bpaabc=cmmib10 at 10.5pt\def\bpi{\hbox{\bpaabc\char'31}}



\pagestyle{headings}

\title{\sl{FEAPpv} - - A Finite Element Analysis Program \\
\rule{4.5in}{1pt} \\ {\large{Personal Version 4.1 User Manual}} \\}



\author{Robert L. Taylor \\
Department of Civil and Environmental Engineering \\
University of California at Berkeley \\
Berkeley, California 94720-1710\\
E-Mail: rlt@ce.berkeley.edu \\}




\date{September 2017}



\begin{document}
\bibliographystyle{unsrt}
\maketitle

\pagenumbering{roman}
\setlength{\parskip}{0mm}
\tableofcontents
\setlength{\parskip}{3mm}

\pagebreak
\pagenumbering{arabic}

\input sect1.tex  % Introduction

\input sect2.tex  % Problem definition

\input sect4.tex  % Manual organization

\input sect5.tex  % Input records

\input sect6.tex  % Mesh input data

\input sect7.tex  % Global Data

\input sect10.tex % Coord and Element Connect

\input sect9.tex  % Coordinate Transformations

\input sect8.tex  % Regions and Groups;

\input sect11.tex % Nodal BC

\input sect12.tex % Surface load

\input sect5il.tex % Include and loop-next

\input sect13a.tex % Elements

\input sect13b.tex % Material models

\input sect14.tex % End and miscellaneous commands

\input sect15.tex % Mesh manipulation

\input sect17a.tex % Command language, time solutions
\input sect17b.tex % Prop,augment,show,hist,eigv etc.

\input sect18.tex % Plot

% Bibliography

%\bibliography{/home/rlt/Feap7/Manual/biblio/book}
\bibliography{/Users/rlt/Feap/biblio/book}

% Appendices

\appendix

\input ../lmesh/appenda.tex
\input ../lmani/appendb.tex
\input ../lmacr/appendc.tex
\input ../lplot/appendd.tex

\end{document}
