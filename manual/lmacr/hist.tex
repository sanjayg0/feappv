\heada{SOLUTION}{HISTory}
\hspace{1.0cm}{{hist,$<$clab,n1,n2$>$ \hfill}}
\headb

The use of the {\tt HIST}ory command permits the user to keep a
history of the previously executed commands and use
this history to reexecute specific commands.  The history
command has several different modes of use which permit easy
control of the execution of commands while in an
interactive mode (use is not recommended in a batch
execution).  The following options are available:

\begin{center}
\begin{tabular}{ l | l | l | l }
clab & n1 & n2 & Description \\ \hline
read &   &   & Input the list of commands   \\
     &   &   & which were 'saved' in a previous   \\
     &   &   & execution.  Warning, this command  \\
     &   &   & will destroy all items currently   \\
     &   &   & in the 'history' list, hence it    \\
     &   &   & should be the first command when   \\
     &   &   & used.                              \\ \hline
save &   &   & Save the previous 'history' of     \\
     &   &   & commands which have been 'added'   \\
     &   &   & to the 'history' list on the file  \\
     &   &   & named 'Feap.hist'.                 \\ \hline
add  &   &   & Add all subsequent commands  \\
     &   &   & executed for the current analysis  \\
     &   &   & to the 'history' list. (default)   \\
noad &   &   & Do not add subsequent commands   \\
     &   &   & executed to the 'history' list   \\ \hline
list & x & x & List the current 'history' of      \\
     &   &   & statements. 'n1' to 'n2',          \\
     &   &   & (default is all in list).          \\
edit & x & x & Delete items 'n1' to 'n2' from     \\
     &   &   & current 'history' list.            \\ \hline
xxxx & x & x & Reexecute commands 'n1' to 'n2'    \\
     &   &   & in the current 'history' list.  \\
     &   &   & (note: 'xxxx' may be anything not  \\
     &   &   & defined above for 'clab' including \\
     &   &   & a blank field.                     \\ 
\end{tabular}
\end{center} 

Use of the history command can greatly reduce the effort in
interactive executions of command language programs.  Since it is not
possible to name the file which stores the history
commands, it is necessary for the user to move any
files needed at a later date to a file other than
{\tt Feap.his} before starting another analysis for which a
history will be retained.  Prior to execution it is necessary
to restore the list to file {\tt Feap.his} before a
{\tt HIST,READ} command may be issued.

Note that the history of commands will not be
saved in {\tt Feap.his} unless a command {\tt HIST,SAVE} is
used.  It is, however, possible to use the history option
without any read or save commands.
\vfill\eject
