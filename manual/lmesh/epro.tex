\heada{MESH}{EPROp}
\hspace{1.0cm}{{ epro                  \hfill}} \\{\smallskip}
\hspace{1.4cm}{{ i-coor,xi-value,(pnum(i),i=1,ndf) \hfill}} \\{\smallskip}
\hspace{1.4cm}{{ <etc.,terminate with a blank record> \hfill}}
\headb

The proportional loading number to be appled to nodal forces and displacments
may be input using this command.
The number may be set along
any set of nodes which has a constant value of the {\it i-coordinate
direction} (e.g., 1-direction (or x), 2-direction
(or  y),  etc.).  The data to be supplied during the definition of
the mesh consists of:

\begin{center}
\begin{tabular}{r l}
\it i-coor   &-- Direction of coordinate (i.e., 1 = x, 2 = y, etc.) \\
\it xi-value &-- Value of i-direction coordinate to be used during  \\
             &\quad search (a tolerance of 1/1000 of mesh size is used  \\
             &\quad during search, any coordinate within the gap is  \\
             &\quad assumed to have the specified value).  \\
\it pnum(1,node) &-- Proportional load number of dof-1 \\
\it pnum(2,node) &-- Proportional load number of dof-2 \\
                 &\quad etc., to {\it ndf} directions \\
\end{tabular}
\end{center}
For nodes with sloping conditions, the degrees-of-freedom
are expressed with respect to the rotated frame 1-2 instead
of the global frame x1-x2 (x-y).  For three dimensional problem
the 3-direction coincides with the x3-direction (z).

Proportional load numbers may also be specified using the {\tt FPRO}p
and {\tt CPRO}p commands.  The data is order dependent with data
defined by {\tt FPRO}p processed first, {\tt EPRO}p processed second and
the {\tt CPRO}p data processed last.  The value defined last is used for
any analysis.
\vfil\eject
