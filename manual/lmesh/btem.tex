\heada{MESH}{BTEMperatures}
\hspace{1.0cm} {{ btem \hfill}} \\{\smallskip}
\hspace{1.4cm} {{ nodes,r-inc,s-inc,t-inc,node1,[r-skip] \hfill}} \\{\smallskip}
\hspace{1.8cm} {{ 1,x1,t1 \hfill}} \\{\smallskip}
\hspace{1.8cm} {{ 2,x2,t2 \hfill}} \\{\smallskip}
\hspace{1.4cm} {{ etc.,until all 'nodes' records are input \hfill}}
\headb

The {\tt BTEM}perature data input segment is used to generate temperatures
on a regular one, two or three dimensional patch
of nodes.  Temperatures specified by {\tt BTEM} command are passed
to the elements in the {\tt tl} array (see programmers manual).
If thermal problems are solved by {\sl FEAPpv} temperatures are {\it generalized
displacements}. Boundary temperatures should then be specified using {\tt DISP},
{\tt EDIS} and/or {\tt CDIS} commands.  Initial conditions are specified using
the {\tt INIT,DISP} solution command.

The temperatures using {\tt BTEM} are generated by
interpolating specified nodal temperatures using the standard
isoparametric interpolation:
$$
T = N_I ( \Bf{\xi} ) \,T_I
$$
where $N_I ( \Bf{\xi} )$ are the shape functions, $\Bf{\xi}$ are the natural
coordinates ($\xi_1$, $\xi_2$, $\xi_3$), and $T_I$ is the temperature at node-I.

For two dimensions, the patch of nodes defined by
{\tt BTEM}perature is developed from a master element which is defined
by an isoparametric 4-9 node mapping function in terms of
the natural coordinates r (for $\xi_1$) and s (for $\xi_2$).
The node numbers on
the master element of each patch defined by {\tt BTEM} are
specified according to Figure \ref{blk2d} in the {\tt BLOC}k manual page.
The four
corner nodes of the master element must be specified, the
mid-point and central node are optional.  For this case {\it t-inc}
is set to 0.

For three dimensions the patch is an 8-27 node brick
where the first 8-nodes are at the corners and the remaining
nodes are mid-edge, mid-face, and interior nodes.  The first
8-nodes must be specified. The block master nodes are numbered as shown in
Figures \ref{blk3da}, \ref{blk3db} and \ref{blk3dc} in the {\tt BLOC}k manual
page.

The data parameters are defined as:

\begin{center}
\begin{tabular}{rl}
\it nodes &-- Number of master nodes needed to define the patch. \\
\it r-inc &-- Number of nodal increments to be generated along \\
          &\quad r-direction of the patch. \\
\it s-inc &-- Number of nodal increments to be generated along \\
          &\quad s-direction of the patch. \\
\it t-inc &-- Number of nodal increments to be generated along \\
          &\quad t-direction of the patch (default = 0). \\
\it node1 &-- Number to be assigned to first node in patch \\
          &\quad (default = 1).  First node is located at same location \\
          &\quad as master node 1. \\
\it r-skip&-- Number of nodes to skip between end of an r-line \\
          &\quad and start of next r-line (may be used to interconnect \\
          &\quad blocks side-by-side) (default = 1)
\end{tabular}
\end{center}
\vfill\eject
